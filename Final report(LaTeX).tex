\documentclass{article}
\usepackage{amsmath}
\usepackage{natbib}
\usepackage{textcomp}

\title{
  \textbf{
    \huge \textbf{Technical University of Munich \\ Campus Straubing \\ Faculty of Bioinformatics}
      }\\
  \vspace{0,25cm}
  \huge Final report: Research internship \\ Applying Reinforcement Learning to \textit{Gym'} Lunar      Lander
  }
\author{Krystian Budkiewicz}
\date{Date: xx.12.2022}
\maketitle
\pagebreak

\begin{document}

%\tableofcontents
\newpage
\section*{Abstract}
This document is the final report summarizing my work during my 6 week research internship at the Faculty of Bioinformatics led by Prof. Dominik Grimm at TUM Campus Straubing. Under the assistance of PhD Student Jonathan Pirnay I was introduced into the basic concepts of Reinforcement Learning (RL) and Deep Q-Learning (DQL). After gaining knowdledge in this topics, i apllied it to solve a toy problem (\textit{Lunar Lander} Box2D environment) using \textit{Python} programming language and \textit{Gym} library, allowing usage of preexisting environments based of Atari 2600 games and other games used for testing DQN-Algorithms. I programmed my own Deep Q-Algorithm which main goal was to solve the environment of \textit{Lunar Lander}, i.e. achieving an average score of 200 in last 100 episodes. This document summarizes what I have learned about both Reinforcement Learning and Deep Q-Learning, and describes their implementation with \textit{Python}.

\newpage
\section*{Introduction}
Sample text Sample text Sample text Sample text Sample text Sample text Sample text Sample text
Sample text Sample text Sample text Sample text Sample text Sample text Sample text Sample text
\\

\section*{Methods}
The main tool used to write the Deep Q-Algorithm was \textis{Python} with its libraries \textbf{PyTorch, Matplotlib, NumPy}, as well as functions \textit{deque()} and \textit{namedtuple()} from library \textbf{collections}
\pagebreak

\section{Reinforcement Learning}
Reinforcement Learning (RL) is a subset of Machine Learning (ML) which is concerned with usage of big datasets for training an AI agent.

RL assumes that every interaction between any agent and an environment can be modeled as maximisation of some kind of reward received by an agent due to interaction with its surroundings. The definition of the reward is ambiguous and its correct decription in code, which is at the hearth of RL reasearch, is the main task of programming using RL.
\subsection{Agent, States, Actions and Rewards}
TXT
\subsubsection{Section}
TXT

\subsection{Bellman equation and loss function}
Bellman equation describes the action-state function $Q^\pi$ based on the current state $s$ and possible actions $a$ to take at that state.

\begin{equation}
\label{eqn:bellman}
Q^\pi(s,a) = r+\gamma Q^\pi(s',\pi(s'))
\end{equation}
\\
Loss function $\mathcal{L}$ is a squared difference of Q-target and the current Q-value.

\begin{equation}
\label{eqn:loss}
\mathcal{L} = \alpha(r+\gamma\max Q(s',a')-Q(s,a))^2
\end{equation}

\section{Deep Q-Learning}
\subsection{Neural Networks - The \textit{Deep} of Q-Learning}
Deep Q-Learning combines the usage of action-state function $Q^\pi$ to describe the state, in which our agent resides, with neural networks processing the input data through a complicated network of layered neurons. At the end, the neural network outputs an "answer" to the current state.

\section{\textit{Lunar Lander} environment}
Until now we considered the technicalities and theory governing the algorithm of an agent. In this section we will go in depth into the implementation of DQN. This part of the research internship consisted of using previously gained theoretical knowdledge of the subject, aswell as online resources (such as tutorials and official documentation of libraries) and applying them in \textit{Python}. Here, using libraries \textit{PyTorch} and \textit{Gym} (as well as other smaller ones, such as e.g. Matplotlib, NumPy), I programmed an Deep Q-Algorithm that was able to interact with a online distributed Box2D environment  \textit{Lunar Lander} and solve it, i.e. achieve an average score of 200 in past 100 episodes.

\subsection{Description of the environment}
In the game of \textit{Lunar Lander} the task of the agent is to land a lunar lander onto an area in the middle of the screen (at coordinates $(0,0)$) in a given time without crashing the vehicle or leaving the bounds set by the environment. The environment is described with an vector with 8 variables (such as x position, y position, angle with respect to the ground, ...). Agent interacts with the environment through 4 actions; doing nothing or turning one of the three engines on (either left, right or bottom).

\subsection{Setting up the environoment}
Using command \textit{import gym} and env = gym.make("LunarLander-v2") the environment is imported into the IDE'\footnotemark \space workspace.
\footnotetext{IDE used was PyCharm Community Edition}

\subsubsection{Reward system}
Reward and the choices of the agent are based on the action-value function \ref{eqn:bellman}.

\subsubsection{Defining the neural network \textit{class}}
The neural network was divided

\newpage
\section*{Results and Discussion}
In the beginning, parameters set during the run (such as learning rate or $\epsilon$) were based on parameters found in code online. After unsuccsessful runs these parameteres were slowly changed in order to improve agents policy. During the first consistent and successful runs the agent was able to solve the environment from 800 to 1200 episodes. Parameters used during these runs are tablified in \ref{tab:x} and will be later referred to as "standard variables".

After the agent could consistently solve the environment in 5 subsequent runs, one of the parameters were changed during the later runs and its influence on the agent, score and loss was noted. In the subsections below detailed influence of these parameters is described.

\subsection*{Behaviours of the agent during training}
In order to solve the environment and maximize the reward from given episode the agent tried multiple tactics. First was hovering with the lunar lander directly or high above the landing area, then slowly descending onto the designated area. Second, dropping fast towards the ground while not breaking the lander, then using one of the engines on the side to push the lander towards the landing area. Third (and the one resembling human-like play-style), slowing down from the initial state and descending in a swing-like motion towards the landing area.

During multiple training cycles the agent showed similar behaviours, which resulted in achieving lower scores, such as not slowing down enough from the starting condition and crashing the lander into the ground, rolling the lander over, not solving the environment in given time (steps), not turning the engines off while on the landing platform or getting out of environments bounds, thus either finishing the episode (in the case of crashing and rolling over) or achieving a huge negative score.

\subsection*{Importing of state dictionary from previous runs}
Importing state dictionaries from neural networks of trained agents to an untrained agent significantly increased the stability of both scores gained by the untrained agent and of the loss function during the run. Also, the agent solved the environment in a much smaller amount of episodes (around 450) then during runs where agent had to learn the environment by itself. The difference between local and target networks casues a huge spike of loss function in the begining of the run, with steady decrease afterwards.

\subsection*{Amount of hidden layers}
The amount of hidden layers influenced the stability of the agent (i.e. the deviation from moving mean score and loss during the run) and lowered the amount of episodes needed until the environment got solved. During the first successful runs where 3 hidden layers were used, the average computation time totaled around 30 minutes and the number of episodes needed to solve the environment variated between 700 and 1100. After that, the amount of hidden layers was gradually increased by 1 and the agent was trained for 5 runs with all other variables remaining the same. Results were tablified in \ref{tab:hidden_layer}.

\begin{table}[h]
\caption{Iterations needed by DQN-Algorithm with specific layer sizes}
\label{tab:table1}
\centering
\begin{tabular}{cc}
  \hline
  \textbf{Size of the layer} & \textbf{Iterations needed} \\
  \hline
  16 & 2 \\
  32 & 6 \\
  64 & 128 \\
  \hline
\end{tabular}
\end{table}

With the change of variables $x, y, z$ following was observed. Change in x did x1.

\begin{table}[h]
\caption{Influence of hidden layers in neural network on training time and episodes needed to solve the environment}
\label{tab:hidden_layer}
\centering
\begin{tabluar}{ccc}
  \hline
  \textbf{Amount of hidden layers} & \textbf{Training time [min]} & \textbf{Last episode}
  \hline
  3 & 30 & ~900 \\
  4 & 20 & ~600 \\
  \hline
\end{tabluar}
\end{table}

Usage of 4 hidden layers resulted in training time of ~20 minutes and number of episodes needed to solve the environment ranged from 500 to 800. Usage of 5 hidden layers resulted in training time of around x minutes and number of episodes needed to solve the environment ranged from

\subsection*{Amount of neurons per hidden layer}
The amount of neurons per hidden layer influenced X. During the first successful runs the amount of neurons per hidden layer equaled 64. After that the amount was gradually changed to powers of 2 beginning at 16 $(16,32,128,256)$. Results were tablified in X.

\subsection*{Epsilon, its minimal value and decay}
Epsilon decay was varied from 0.995, 0.9975 and 0.999 (all other parameters standard, amount of hidden layers set to 3), with every run beginning with the $\epsilon$-value of 1.

\end{document}